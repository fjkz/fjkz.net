\documentclass[a4paper,11pt]{article}
\usepackage{latexsym}
\usepackage{amsmath}
\usepackage{amssymb}
\usepackage{bm}
\usepackage{color}

\setlength{\textheight}{40\baselineskip}
\setlength{\topmargin}{35mm}
\addtolength{\topmargin}{-1.5in}    

\setlength{\textwidth}{150mm}
\setlength{\oddsidemargin}{30mm}
\addtolength{\oddsidemargin}{-1in}  
\setlength{\evensidemargin}{30mm}
\addtolength{\evensidemargin}{-1in} 

\setlength{\footskip}{30mm}

\pagestyle{plain}

\renewcommand{\figurename}{Fig.}
\renewcommand{\tablename}{Table}

\title{The optimum ratio of devices in tiered storage}
\author{Kazuho Fujii}
\date{Mar. 9, 2015}

\begin{document}
\maketitle

\section{Background}
Storage tiering is a technique to create a high performance and cheap storage by combining several kinds of storage devices,
in which the more accessed data is allocated to the higher performance device and the fewer accessed data is allocated to the cheaper device.
For example, SSD, SAS HDD and SATA HDD are usually used for each tiers.

However, it is unclear how we should decide the ratio of devices or what kinds of devices should be combined for efficiency.
If we can calculate the optimum ratio of devices by the access pattern, we can effectively allocate devices from storage pools.
In this report, I derive the optimum ratio of devices that maximizes the performance under some assumptions:
we can decide uniquely storage device performance and cost per a unit capacity;
and the access pattern obeys the Pareto distribution. Additionally I discuss about the condition of devices for effective tiering.

\section{Derivation of the optimum ratio of devices}

I think about creating tiered storage with $N$ kinds of recording devices: Device-1, Device-2, $\cdots$, Device-$N$.
I express the performance value of Device-$i$ as $q_i$ and express the cost per a unit capacity as $c_i$, where  $q_1<q_2<\cdots<q_N$, $c_1<c_2<\cdots<c_N$.
I express the ratio of Device-$i$ capacity to the total capacity of a volume as $r_i$, where
\begin{eqnarray}
\sum_{i=1}^N r_i = 1 \label{sum_ratio}\\
r_i > 0
\end{eqnarray}
. The mean cost per a unit capacity $\bar{c}$ is
\begin{eqnarray}
\bar{c}&=&\sum_{i=1}^N r_i c_i \label{mean_cost}
\end{eqnarray}
.

I think about the case where the mean cost $\bar{c}\ (c_1 \leq \bar{c} \leq c_N)$ is provided as a budget.
Now I want to derive the ratio of each drives $r_1, r_2, \cdots, r_N$ which maximizes the mean performance value $\bar{q}$ under the constraints \eqref{sum_ratio}\eqref{mean_cost}.
I can use the method of Lagrange multiplier for this problem.
With multiplier $\mu$ and $\lambda$, I define a function $g$ as
\begin{eqnarray}
g&:=&\bar{q} - \lambda \left( \bar{c} - \sum_{i=1}^{N} r_i c_i \right) 
-\mu \left( 1- \sum_{i=1}^N r_i \right)
\end{eqnarray}
.
I can obtain $r_1, r_2, \cdots ,r_N$ where $\bar{q}$ get an extreme value by solving the simultaneous equations of \eqref{sum_ratio}\eqref{mean_cost} and $N$ equations below:
\begin{eqnarray}
\frac{\partial g}{\partial r_i} &=& 0
  ,\ (i = 1, 2, \cdots , N)
  \label{lagrange_theory}
\end{eqnarray}
.

However, $\bar{q}$ is unknown. In order to solve the equations, I have to express $\bar{q}$ with an equation including $r_i$.
Here I assume that the distribution of access to storage sectors or blocks (written just with sectors below) obeys the Pareto distribution, sometimes called the law of 80:20.
That is we can express the probability $F$, where the number of access to a sector is not more than $x$, with the equation below: 
\begin{eqnarray}
F=1-\left(\frac{x_0}{x}\right)^a,\ (a > 1,\ x_0 > 0)
\end{eqnarray}

$a$ and $x_0$ are parameters of the probability distribution.
$a$ means bias of access, so called Pareto coefficient.
As $a \approx 1$ access is concentrated to a sector;
as $a \to \infty$ access is uniformly distributed, that is the number of access at each sector is $x_0$. The probability density function $f$ is 
\begin{eqnarray}
f &:=& \frac{\partial F}{\partial x}\nonumber\\
  &=& \frac{a x_0^a}{x^{a+1}}
\end{eqnarray}
.

I assume that the tiering is perfect; the higher accessed sector is allocated to the higher performance device for each sector. 
I allocate sectors where access count $x$ is $x_{i-1} \leq x < x_i$ to Device-$i$; where $x_N = \infty$.
The relationship between $r_i$ and $x_i$ is
\begin{eqnarray}
r_1+r_2+\cdots+r_i=1-\left(\frac{x_0}{x_i}\right)^a \\
x_i=x_0\left(1-r_1-r_2-\cdots-r_i\right)^{-\frac{1}{a}}
\end{eqnarray}
. The probability of access to a sector on Device-$i$ $p_i$ is
\begin{eqnarray}
p_i &=& \frac{\displaystyle{\int_{x_{i-1}}^{x_i} x f dx}}{\displaystyle{\int_{x_0}^{\infty} x f dx}} \nonumber\\
&=& \left(1-r_1-\cdots-r_{i-1}\right)^\frac{a-1}{a}
   -\left(1-r_1-\cdots-r_i\right)^\frac{a-1}{a} 
\end{eqnarray}
.
The mean performance $\bar{q}$ is
\begin{eqnarray}
\bar{q}&=&\sum_{i=1}^N p_i q_i \nonumber\\
&=&\sum_{i=1}^N \left\{ \left(1-r_1-\cdots-r_{i-1}\right)^\frac{a-1}{a}
                 -\left(1-r_1-\cdots-r_i    \right)^\frac{a-1}{a}
\right\} q_i \label{mean_quality}
\end{eqnarray}
.

By solving the simultaneous equations \eqref{sum_ratio}\eqref{mean_cost}\eqref{lagrange_theory}\eqref{mean_quality},
I can get the optimum ratio of each drive $r_1,r_2,\cdots,r_N$ where the mean performance $\bar{q}$ get the maximum value; that is 
\begin{eqnarray}
r_1&=&(\bar{c}-c_1)\left\{
    \sum_{i=1}^{N-1}\Delta c_i \left(\frac{\Delta c_i}{\Delta q_i}
    \right)^{-a}\right\}^{-1}
\left\{ 
-     \left(\frac{\Delta c_1}{\Delta q_1}
    \right)^{-a}
\right\}+1 \label{r1_c}\\
r_i&=&(\bar{c}-c_1)\left\{
    \sum_{i=1}^{N-1}\Delta c_i \left(\frac{\Delta c_i}{\Delta q_i}
    \right)^{-a}\right\}^{-1}
\left\{ 
     \left(\frac{\Delta c_{i-1}}{\Delta q_{i-1}}
    \right)^{-a}
-     \left(\frac{\Delta c_i}{\Delta q_i}
    \right)^{-a}
\right\}, \nonumber \\
&& (i = 2, 3, \cdots, N-1)\label{ri_c}\\
r_N&=&(\bar{c}-c_1)\left\{
    \sum_{i=1}^{N-1}\Delta c_i \left(\frac{\Delta c_i}{\Delta q_i}
    \right)^{-a}\right\}^{-1}
\left\{ 
    \left(\frac{\Delta c_{N-1}}{\Delta q_{N-1}}
    \right)^{-a}
\right\} \label{rN_c}
\end{eqnarray}
\begin{eqnarray}
\Delta c_i &:=& c_{i+1} - c_i \\
\Delta q_i &:=& q_{i+1} - q_i
\end{eqnarray}
. In this case, $\bar{q}$ is
\begin{eqnarray}
\bar{q} = q_1 + \left(\bar{c}-c_1\right)^\frac{a-1}{a}
\left\{\sum_{i=1}^{N-1} \Delta c_i
 \left(\frac{\Delta c_i}{\Delta q_i}\right)^{-a}
\right\}^{-\frac{a-1}{a}} 
\left\{\sum_{j=1}^{N-1}
 \Delta q_j  \left(\frac{\Delta c_j}{\Delta q_j}\right)^{-a+1}
\right\}
\end{eqnarray}
.

\section{Discussion}

In the parameters of the assumed probability distribution, $x_0$ doesn't appear but just the Pareto coefficient $a$ appears.
Therefore, for getting the optimum ratio we have to assume only $a$, which differs in situations where storage used. We can get the Pareto coefficient statistically.

The condition where the optimum ratio of Device-$i$ $r_i > 0\ (i = 2, 3, \cdots, N-1)$  is
\begin{eqnarray}
\frac{\Delta c_{i-1}}{\Delta q_{i-1}} < \frac{\Delta c_i}{\Delta q_i}
\label{satuation}
\end{eqnarray}
. This means points $(c_i, q_i)$ are on a convex curve in a graph where $x$ and $y$ axes are respectively cost and performance.
We can not enhance mean performance or reduce mean cost if we combine drives that do not meet this condition. 
The combination of SSD, SAS HDD and SATA HDD could not meet this condition bacause SSD became cheaper.

The condition where the ratio of Device-1 $r_1 > 0$ is

\begin{eqnarray}
\bar{c} < c_1 + \left(\frac{\Delta c_1}{\Delta q_1}\right)^a \sum_{i=1}^{N-1}\Delta c_i
 \left(\frac{\Delta c_i}{\Delta q_i}\right)^{-a} \label{over_budget}
\end{eqnarray}
.This means that the lowest performance device has no effect when having enough budget.
In this case we should apply \eqref{r1_c}\eqref{ri_c}\eqref{rN_c} with removing the lowest performance device.

\section{Conclusion}

In this report the optimum ratio of devices are derived that maximize a tiered storage performance.
It can be applied to designing tiered storage or developing a software that automatically adjusts tiering system configurations.
It is under consideration for storage tiering, but it could be also applied to caching.

Additionally the fact is cleared that kinds of devices in tiers require some conditions for effective performance.
Some tiered storage systems might take a penalty by unsatisfying the conditions.

\appendix

\section{In the case provided a performance requirement}

In the case where a required performance value $\bar{q}\ (q_1 \leq \bar{q} \leq q_N)$ is provided, I can get the optimum ratio with the same method.
In this case I want the ratio $r_1,r_2,\cdots,r_N$ which minimize mean costs $\bar{c}$ under the constraints \eqref{sum_ratio}\eqref{mean_quality}.

The result is below:
\begin{eqnarray}
r_1 &=& \left(\bar{q}-q_1\right)^\frac{a}{a-1}
\left\{\sum_{i=1}^{N-1}\Delta q_i\left(\frac{\Delta c_i}{\Delta
				q_i}\right)^{-a+1}\right\}^{-\frac{a}{a-1}}
\left\{-\left(\frac{\Delta c_1}{\Delta q_1}\right)^{-a}\right\} +
1 \label{r1_q} \\
r_i &=& \left(\bar{q}-q_1\right)^\frac{a}{a-1}
\left\{\sum_{i=1}^{N-1}\Delta q_i\left(\frac{\Delta c_i}{\Delta
				q_i}\right)^{-a+1}\right\}^{-\frac{a}{a-1}}
\left\{\left(\frac{\Delta c_{i-1}}{\Delta q_{i-1}}\right)^{-a}
-\left(\frac{\Delta c_i}{\Delta q_i}\right)^{-a}\right\}
, \nonumber \\
&& (i = 2, 3, \cdots, N-1) \label{ri_q}\\
r_N &=& \left(\bar{q}-q_1\right)^\frac{a}{a-1}
\left\{\sum_{i=1}^{N-1}\Delta q_i\left(\frac{\Delta c_i}{\Delta
				q_i}\right)^{-a+1}\right\}^{-\frac{a}{a-1}}
\left\{\left(\frac{\Delta c_{N-1}}{\Delta
	q_{N-1}}\right)^{-a}\right\}
\label{rN_q}
\end{eqnarray}
. In this case,
\begin{eqnarray}
\bar{c} = c_1 + \left(\bar{q}-q_1\right)^\frac{a}{a-1}
\left\{\sum_{i=1}^{N-1}\Delta q_i\left(\frac{\Delta c_i}{\Delta
				q_i}\right)^{-a+1}\right\}^{-\frac{a}{a-1}}
\left\{\sum_{j=1}^{N-1}\Delta c_j \left(\frac{\Delta c_{j}}{\Delta q_{j}}\right)^{-a}\right\}
\end{eqnarray}
.

The required performance $\bar{q}$ needs the condition below:
\begin{eqnarray}
\bar{q} &<& q_1 + 
\left(\frac{\Delta c_1}{\Delta q_1}\right)^{a-1}
\sum_{i=1}^{N-1}\Delta q_i\left(\frac{\Delta c_i}{\Delta
				q_i}\right)^{-a+1} 
\end{eqnarray}
. When the required performance is over this value, we can not include the lowest performance device into tiers.
In this case, we should apply \eqref{r1_q}\eqref{ri_q}\eqref{rN_q} with removing the device.

However, this result may not satisfy the required performance because assuming perfect tiering.

\section{In the case considering tiering efficiency}

I think about the case when tiering is not perfect.
I express the probability with $s\ (0 < s\leq 1)$ where data exists at the best device. In this case, the probability of access to a sector on Device-$i$ $\tilde{p}_i$ is
\begin{eqnarray}
\tilde{p}_i = sp_i + (1-s)r_i
\end{eqnarray}
The mean performance $\tilde{q}$ is 
\begin{eqnarray}
\tilde{q} &=& \sum_{i=1}^N\tilde{p}_i q_i \nonumber\\
&=& \sum_{i=1}^N \left[s \left\{
 \left(1-r_1-\cdots-r_{i-1}\right)^\frac{a-1}{a}-\left(1-r_1-\cdots-r_i\right)^\frac{a-1}{a}
\right\}
 + (1-s) r_i\right] q_i
\end{eqnarray}
. However, I can not analytically get the ratio of devices $r_1,r_2,\cdots,r_N$ which maximizes $\tilde{q}$. Thus a numerical analysis technique is required.

Now I need a complex number $\lambda$ that satisfies the equation below:
\begin{eqnarray}
h(\lambda) :=
\bar{c}-c_1 - \left(\frac{s(a-1)}{a}\right)^{a}
\sum_{i=1}^{N-1}\Delta c_i \left(
\lambda\frac{\Delta c_i}{\Delta q_i} + 1-s
\right)^{-a} =
0 \label{h=0} 
\end{eqnarray}
I can solve it with the Newton method;
$\lambda_n$ becomes $\lambda$ that meets \eqref{h=0} as repeating the operation below from a proper initial value $\lambda_0$:
\begin{eqnarray}
\lambda_{n+1} &=& \lambda_n - \frac{h(\lambda_n)}{h'(\lambda_n)}
\end{eqnarray}
If the initial value is not proper, $\lambda_n$ do not convergence.
The solution of \eqref{h=0} is obvious when $s = 1$. 
As closing $s$ to the wanted value, solving \eqref{h=0} for each $s$ with using previous $s$ as an initial value,
I could conduct stable calculation.

With the solution $\lambda$, the ratio of devices $r_1,r_2,\cdots,r_N$ which maximizes $\tilde{q}$ is
\begin{eqnarray}
r_1&=&\left(\frac{s(a-1)}{a}\right)^{a}
\left\{ 
     \left(\lambda\frac{\Delta c_1}{\Delta q_1}+1-s
    \right)^{-a}
\right\}+1 \label{r1_cs}\\
r_i&=&\left(\frac{s(a-1)}{a}\right)^{a}
\left\{ -
     \left(\lambda\frac{\Delta c_{i-1}}{\Delta q_{i-1}}+1-s
    \right)^{-a}
+     \left(\lambda\frac{\Delta c_i}{\Delta q_i}+1-s
    \right)^{-a}
\right\}, \nonumber \\
&& (i = 2, 3, \cdots, N-1)\label{ri_cs}\\
r_N&=&\left(\frac{s(a-1)}{a}\right)^{a}
\left\{-
    \left(\lambda\frac{\Delta c_{N-1}}{\Delta q_{N-1}}+1-s
    \right)^{-a}
\right\} \label{rN_cs}
\end{eqnarray}

\end{document}
